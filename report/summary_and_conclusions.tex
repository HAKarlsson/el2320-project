\section{Summary and Conclusions}
We compared four different particle filter resampling methods, Systematic Resampling, two versions of Cluster Resampling and Multi Particle Filter (MPF) Resampling. The comparison was made by tracking ants on a video feed. The MPF Resampling method performed the best were 50-80\% of the ants were detected. In the second place were Cluster Resampling using Lloyd's algorithm to form clusters, it detected 40-50\% of the ants. The second variant using EM for clustering performed worse, detecting about 20\% of the ants. The absolute worst method was Systematic Resampling, detecting only one ant.

The MPF Resampling have some issues, it suffers from having negligible sub particle filters and thus ended up with particles in locations where no ants are present. This was partially solved by having a threshold $\tau$ such that any sub particle filter with normalized weight less than $\tau$ were removed. Setting the value $\tau$ is difficult, setting it too high can decimate the number of particle filters, and if it is too low, it does not remove enough particle filters.

The Cluster Resampling methods were moderately good with no negligible particles as with the MPF method but they were considerably slower. The Lloyd's version of clustering was both faster and more accurate than the EM variant. The reason it is more accurate might be due to the EM variant will always assign at least one particle to a cluster, therefore it might end up splitting one cluster into many smaller clusters that becomes relatively negligible in size. A solution might be to partially assign particles to all clusters using the EM algorithm and thereafter resampling is done cluster by cluster using normalized partial assigned weights; that is, the weight used for cluster $n$ is $c_{i,n}w_t^i$ where $c_{i,n}$ is the partial assignment of particle $i$ to cluster $n$.

The Cluster Resampling method might improve if a constant number of particles were kept for each cluster as it now depends on the total weight of the cluster. However, if the size is constant we might get problems with negligible clusters. The solution could be a threshold for each cluster as in MPF Resampling. 

The Cluster Resampling for the ant dataset performed notably worse what \cite{vermaak2003maintaining} used for tracking football players, this is probably due to that the ants were tracked in gray scale so they blend in with the background, and there were 20 ants tracked compared to 4 football players. Other things were that we used a simpler version of the algorithm where cluster weights were neglected.