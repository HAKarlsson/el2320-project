\documentclass{article}
\usepackage[margins=2in, a4paper]{geometry}

\begin{document}
	\noindent
	{\Large \textbf{EL2320 Project Proposal}}
	\vspace{0.5cm}
	
	\noindent
	\begin{tabular}{ll}
		\textbf{Name:}  & Henrik Karlsson\\
		\textbf{Email:} & \texttt{henrik10@kth.se}
	\end{tabular}
	\vspace{0.5cm}
	
	\noindent
	\paragraph{Stratified particle filter and EKF mixture models} I will to implement stratified particle filter and mixture of EKF in Matlab, then test them on the datasets given by the labs. Once that is done I want to combine the approaches by allowing mixtures to alternate between EKF and particle filter depending on the variance in EKF/particles. Hopefully, this can result in faster computations. 
	
	The idea of the combined approach is that a particle filter will transform into an EKF if the distribution of particles can be modeled with an EKF, and that an EKF can turn into a particle filter when the distribution is not Gaussian (not oval enough, multimodal). Some of the immediate problems that needs to be addressed when implementing the idea are the following:
	\begin{itemize}
		\item When to switch between EKF and particle filter
		\item How to switch between EKF and particle filter (number of particles, mean, weights, covariances, etc.)
		\item Should we switch to particle filter or split into two EKF mixtures
	\end{itemize}
	
	If I have time I will implement the EKF/Particle filter with adaptive number of particles or groups/mixtures.
	
\end{document}